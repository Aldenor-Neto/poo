\documentclass[12pt]{article}

\usepackage[utf8]{inputenc}
\usepackage{setspace}
\usepackage{lipsum}

\usepackage[brazil]{babel}   
\usepackage[T1]{fontenc}

\title{Filtro de conteúdo acessível dentro do youtube para deficiente visual}
\author{SILVA NETO, F. A. \thanks{Estudante de Ciências da Computação do IFCE campus Maracanaú. Pesquisador Júnior em IA no projeto SIAS (setembro de 2023 a agosto de 2024) pelo Polo de Inovação e Pesquisa Embrapi. Pesquisador Júnior em IA no projeto CCLI (agosto de 2024 a julho de 2026) também pelo Polo Embrapi. Analista de Acessibilidade no projeto NEES (julho de 2023 a junho de 2025) pela Fundação Universitária de Desenvolvimento e Pesquisa da UFAL. Email: aldenor.neto02@aluno.ifce.edu.br}}

\date{\vspace{-5ex}}

\begin{document}

\maketitle

\begin{abstract}
\onehalfspacing
O BlindTube\thanks{Nome da aplicação de filtro de conteúdo acessivél na plataforma do Youtube.} é uma solução inovadora para as necessidades de acessibilidade de pessoas com deficiência visual no YouTube. O aplicativo simplifica a busca por conteúdo e prioriza canais comprometidos com a inclusão, oferecendo recursos como leitores de tela e transcrições de fala para texto.
Encontrar conteúdo acessível é desafiador para pessoas com deficiência visual, devido à variabilidade na oferta de legendas e outras ferramentas de acessibilidade entre os canais. O BlindTube resolve isso ao destacar canais que adotam práticas inclusivas, tornando a navegação mais eficiente e satisfatória.
O aplicativo se destaca por funcionar em dispositivos móveis e desktops, garantindo uma experiência consistente e acessível em diferentes plataformas. Isso permite que usuários com deficiência visual acessem facilmente o conteúdo necessário, independentemente do dispositivo.
Além de facilitar a navegação, o BlindTube também sensibiliza para a importância da acessibilidade na criação de conteúdo online. Ao valorizar canais que investem em recursos acessíveis, incentiva outros criadores a adotarem práticas semelhantes, promovendo um ambiente mais inclusivo no YouTube.
Em resumo, o BlindTube facilita o acesso de pessoas com deficiência visual e contribui para uma comunidade online mais acessível. Como plataforma que prioriza a inclusão, o aplicativo exemplifica como a tecnologia pode promover a igualdade no acesso ao conteúdo digital.
 
\textbf{Palavras-chave:} Acessibilidade, Inclusão, Deficiencia Visual, aplicativo acessivel, Leitor de tela.
\end{abstract}

\begin{abstract}
\onehalfspacing
BlindTube is an innovative solution for the accessibility needs of visually impaired individuals on YouTube. The app simplifies the search for content and prioritizes channels committed to inclusion, offering features such as screen readers and speech-to-text transcriptions.
Finding accessible content is challenging for visually impaired people due to the variability in the availability of captions and other accessibility tools across channels. BlindTube addresses this by highlighting channels that adopt inclusive practices, making navigation more efficient and satisfying.
The app stands out for working on both mobile devices and desktops, ensuring a consistent and accessible experience across different platforms. This allows visually impaired users to easily access the content they need, regardless of the device.
In addition to facilitating navigation, BlindTube also raises awareness about the importance of accessibility in online content creation. By valuing channels that invest in accessible resources, it encourages other creators to adopt similar practices, promoting a more inclusive environment on YouTube.
In summary, BlindTube makes it easier for visually impaired individuals to access content and contributes to a more accessible online community. As a platform that prioritizes inclusion, the app exemplifies how technology can promote equality in access to digital content.
 

\textbf{Keywords:} BlindTube, Accessibility, Inclusion, Visual Impairment, Blind.
\end{abstract}

\section{INTRODUÇÃO}

A acessibilidade nos meios digitais para pessoas com deficiência visual ainda é bastante limitada até os dias de hoje. Apesar dos avanços significativos alcançados com o desenvolvimento de leitores de tela, tanto na área mobile (telefonia) quanto na área web (desktop), ainda há uma carência significativa nos recursos disponibilizados, o que resulta em dificuldades de navegação para os deficientes visuais nesses ambientes.
No que diz respeito à plataforma de streaming de vídeo da Google, conhecida como YouTube, uma das principais dificuldades enfrentadas pelos usuários com deficiência visual está relacionada à falta de descrição das cenas ou acontecimentos presentes nos vídeos. Muitas vezes, os desenvolvedores de conteúdo não fornecem detalhes sobre suas ações durante o vídeo, tornando praticamente impossível para pessoas com deficiência visual aproveitarem o conteúdo de forma adequada.
Essa lacuna na acessibilidade do YouTube pode ser especialmente frustrante para os deficientes visuais, pois priva-os da experiência completa e impede seu pleno engajamento com o conteúdo. 
 
\section{Desenvolvimento}

\section{Acessibilidade em aplicações mobile e web}

A acessibilidade em aplicações, seja mobile ou desktop, ou até mesmo a falta de acessibilidade em tais recursos, se dá pelo fato de que, na maioria das vezes, os desenvolvedores de software não se preocupam com esse quesito. Em diferentes linguagens de marcação, tais como os frameworks disponibilizam em sua estrutura recursos que possibilitam que pessoas com deficiência visual também tenham acesso ao conteúdo. No entanto, na maioria das vezes, seja por falta de conhecimento ou até mesmo por comodismo, esses desenvolvedores não se atentam a esse detalhe crucial de sua aplicação. Parâmetros como "Aria-label" de uma tag "button" do HTML possibilitam que o usuário que faz uso do leitor de tela possa saber do que se trata tal botão. Caso contrário, se não for acrescentado o atributo, o leitor de tela fará a leitura apenas como "botão" em vez de ler o título do mesmo. 

``(...) Observa-se que 74\% das contribuições estão relacionadas ao processo de desenvolvimento de software.(...)`` \cite{semish2016}.


\section{Surgimento da aplicação BlindTube}

Com base nessa realidade enfrentada pelas pessoas com deficiência visual e visando facilitar a busca por material acessível quando solicitado pelo PCD\thanks{Abreviação para pessoa com deficiencia} visual, o BlindTube desenvolveu um filtro de pesquisa. Esse filtro tem o objetivo de exibir ao usuário com deficiência o material solicitado, porém com recursos de acessibilidade, proporcionando-lhe uma experiência mais satisfatória ao navegar no YouTube e evitando que ele perca tempo procurando vídeos relevantes.

Para ilustrar a proposta do BlindTube, considere um usuário com deficiência visual realizando uma busca simples na plataforma do YouTube, 
buscando por ``Curso de Windows''. Normalmente, o YouTube apresentaria diversos resultados relacionados, mas dificilmente incluiria cursos que sejam compatíveis com leitores de tela, 
como o NVDA (\textit{NonVisual Desktop Access}) ou JAWS (\textit{Job Access With Speech}), entre outros. No entanto, ao utilizar o BlindTube, o resultado da busca será significativamente diferente. Antes de exibir os resultados, o BlindTube realiza um filtro específico para ``Cursos de Windows'' que fazem uso de leitores de tela, descrições de imagens e conteúdo visual, como transcrições em texto do vídeo. Essas pequenas diferenças mencionadas no exemplo proporcionam uma experiência mais enriquecedora para as pessoas com deficiência visual ao navegarem pelos meios digitais, como é o caso da navegação pelo YouTube.
A acessibilidade é crucial no contexto atual, pois reconhecemos a diversidade das necessidades humanas em ambientes físicos e virtuais. No Brasil, disciplinas de Interação Humano-Computador (IHC) frequentemente discutem Design Inclusivo e Acessibilidade. No entanto, desafios persistem na inclusão de pessoas com deficiência visual. A Acessibilidade na Web é fundamental para garantir que todos, incluindo pessoas com deficiência visual, tenham acesso igual ao conteúdo online \cite{proedad2023}.

\section{Estrutura do software}

A aplicação foi desenvolvida utilizando \textit{Spring Boot} para o \textit{back-end} (\textit{back-end} refere-se à parte do sistema que processa dados e lógica de negócios), uma estrutura que simplifica o desenvolvimento de aplicativos. O código do \textit{back-end} foi escrito em \textit{Java}, uma linguagem de programação amplamente reconhecida por sua robustez e portabilidade.

Para gerenciar as dependências do projeto e facilitar a compilação, empacotamento e gestão de bibliotecas, foi utilizado o \textit{Maven}, uma ferramenta de automação de compilação e gerenciamento de dependências.

No \textit{front-end} (\textit{front-end} é a interface direta com a qual os usuários interagem), a escolha recaiu sobre o \textit{Angular}, um \textit{framework} JavaScript/\textit{TypeScript} desenvolvido pelo Google. \textit{Angular} foi selecionado por sua capacidade de criar interfaces de usuário interativas e responsivas em aplicativos web.

Além disso, a aplicação segue o padrão \textit{MVC} (\textit{Model-View-Controller}), o que facilita a organização e manutenção do código, garantindo uma separação clara entre a lógica de negócios, a apresentação dos dados e a interação do usuário.

\section{Conclusão}

Portanto, concluímos que o Blindtube, apesar de aparentemente ser uma aplicação simples, trará um grande desempenho e praticidade para a vida da pessoa com deficiência visual. Seu uso contribuirá para que o PCD visual tenha mais tempo para realizar outras atividades, em vez de desperdiçá-lo em busca de conteúdo relevante na plataforma de streaming do YouTube.
Além disso, é importante ressaltar que o Blindtube não apenas oferece eficiência na busca por conteúdo, mas também promove uma maior inclusão digital ao proporcionar acessibilidade para os usuários com deficiência visual. Isso não só amplia suas oportunidades de entretenimento e informação, mas também fortalece sua participação na era digital.

\bibliographystyle{plain}
\bibliography{referencias} 
\end{document}
